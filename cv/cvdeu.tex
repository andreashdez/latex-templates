%%
%% Lebenslauf in TeX
%%
%% Layout for a German style CV (Lebenslauf) using the
%% currvita package. It has been developed to be compiled
%% with LuaLaTeX (tested on version 1.0.4).
%% !TEX program = lualatex
%%
%% Copyright (C) 2018 Andreas Hernández Hauser
%%

\documentclass[12pt,a4paper,oneside,final]{scrartcl}

\usepackage{cvdeu}

\begin{document}

\noindent\textbf{\Large{Lebenslauf}}

\vspace{1em}

\noindent{\LARGE{Andreas G. Hernández Hauser}}

\vspace{2em}

\begin{cv}

    \begin{cvlist}{Persönliche Daten}
        \setlength\itemsep{-0.5em}
        \item[Name] Andreas G. Hernández Hauser
        \item[Anschrift] Straße~000\\
            00000~Stadt
            \setlength\itemsep{0em}
        \item[Mobil] 0000~0000000
        \setlength\itemsep{-0.5em}
        \item[E-Mail] \href{mailto:email@email.com}{email@email.com}
        \setlength\itemsep{0em}
        \item[Geburtsdatum/-ort] 00. Januar 1900 in~Stadt
        \setlength\itemsep{-0.5em}
        \item[Staatsangehörigkeit] deutsch
        \item[Familienstand] ledig

        \sepspace
    \end{cvlist}

    \begin{cvlist}{Ausbildung}
        \item[08/1995--05/2000] Deutsche Schule in Puerto de la Cruz, Teneriffa, Spanien\\
            \textit{Unterricht~auf~Deutsch~und~Spanisch}
        \item[08/2000--05/2007] International British Yeoward School in Puerto de la Cruz\\
            \textit{Unterricht~auf~Englisch}\\
            Abschluss: A-level (Abitur)
        \item[08/2009--04/2013] \textbf{Grado en Ingeniería Informática}\\
            Universidad de La Laguna, Teneriffa, Spanien
        \item[08/2013--05/2016] \textbf{Computer Science \& A.I.}\\
            Aberystwyth University, Wales, UK\\
            Abschluss: Bachelor of Science

            \textit{Relevante Module:}\\[0.5em]
            \textbf{Concepts in Programming} und \textbf{Software Development:} Einführung in die Programmierung mit unterschiedlichen Konzepten. Objektorientierung mit Java und funktionale Programmierung mit Haskell.\\[0.5em]
            \textbf{Program Design, Data Structures and Algorithms:} Codierung von Datenstrukturen mit Java (Warteschlange, Stapelspeicher, Binärer Heap) und Dijkstra-Algorithmus.\\[0.5em]
            \textbf{C and UNIX Programming:} Einführung in die C Programmierung, Zeiger und Datenstrukturen.\\[0.5em]
            \textbf{C, C++ and Java Programming Paradigms:} Unterschiede zwischen objektorientierter und prozeduraler Programmierung. Detaillierte Einführung in die C++ Programmierung und Templates. Objektorientierte Programmierung mit C (Funktionszeiger in Strukturen).

            \textit{Titel der Bachelorarbeit:}\\[0.5em]
            \textbf{Multiclass image classification with convolutional neural networks}\\
            Klassifizierung der Klassen des CIFAR-10 Datensatzes mit ConvNets unterschiedlicher größen und Parametern.

        \item[08/2016--09/2017] \textbf{Software Engineering (Computer Science)}\\
            Aberystwyth University, Wales, UK\\
            Abschluss: Master of Science

            \textit{Relevante Module:}\\[0.5em]
            \textbf{The Objectoriented Programming Paradigm:} Einschätzung einer Auswahl von Problemdomänen für eine angemessene Anwendung der Objectorientierung. Anwendung von unterschiedlichen Entwurfsmustern (design patterns) und Auswertung dessen effektivität.\\[0.5em]
            \textbf{Fundamentals of Intelligent Systems:} Einführung in unterschiedlichen Konzepten der Künstlichen Intelligenz und deren Implementierung in Python und Java.\\[0.5em]
            \textbf{Machine Learning for Intelligent Systems:} Einführung in unterschiedlichen Konzepten für Machine Learning. Analyse von Datensätzen mit Python.

            \textit{Titel der Masterarbeit:}\\[0.5em]
            \textbf{Development of computational tools for image-based phenotyping of Arabidopsis}\\
            Verwendung traditioneller Bildsegmentierungstechniken zum erkennen verschiedener Teile der Arabidopsis-Pflanze. Einsatz von Deep-Learning-Techniken mit verschiedenen ConvNets zur Objekterkennung.

        \sepspace
    \end{cvlist}

    \begin{cvlist}{Sonstiges}
        \item[10/2007--06/2008] \textbf{Wehrdienst bei der Bundeswehr in Deutschland (Luftwaffe)}\\
            u. a. Ausbildung als Helfer im Sanitätsdienst und Lehrgang für Logistikeinheiten
        \item[2004--2017] \textbf{Buchführung für Wohnungseigentümergemeinschaften}\\
            (Nebenberuflich) Buchführung und Bilanzierung sowohl mittels spezieller Computerprogramme als auch mittels Tabellenkalkulation

        \sepspace
    \end{cvlist}

    \begin{cvlist}{Kenntnisse}
        \item[Betriebssysteme] Linux, Windows, macOS
        \item[Sprachen] Java, C/C++, Python, Shell-Skripte, HTML, JavaScript
        \item[Anwendungen] Eclipse, IntelliJ IDEA, Visual Studio, Android Studio, LaTeX, Vim
        \item[Sprachkenntnisse] Deutsch und Spanisch \textit{(Muttersprache)}\\
            Englisch \textit{(Englische Schule in Puerto de la Cruz und Studium im Vereinigten Königreich)}\\
            Französisch \textit{(Grundkenntnisse)}

        \sepspace
    \end{cvlist}

    \begin{cvlist}{Interessen}
        \item Open-Source, Programmierung, GNU/Linux,\\
            Machine Learning, Künstliche neuronale Netze,\\
            Klassische Musik, Karate, Typografie
    \end{cvlist}

    \cvplace{Düsseldorf}
    \date{15.~März~2018}

\end{cv}

\end{document}
